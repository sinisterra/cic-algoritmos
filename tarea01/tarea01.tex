\documentclass[hidelinks]{article}
\usepackage[spanish]{babel}
\usepackage{hyperref}
\newcommand\Omicron{O}
\newcommand{\br}{\vskip.2cm \noindent}

\newtheorem{propiedad}{Propiedad}[section]

\author{Santiago Sinisterra Sierra}
\title{Diseño y Análisis de Algoritmos \\ Tarea 1 - Demostraciones de las propiedades de los órdenes asintóticos}
\date{\today}

\begin{document}

\maketitle

\section{Transitividad}

\begin{propiedad}
  Si $f(n)$ es $\Omicron(g(n))$ y $g(n)$ es $\Omicron(h(n))$ $\Rightarrow$ $f(n)$ es $\Omicron(h(n))$
\end{propiedad}

\br $f(n)$ es $\Omicron(g(n))$ es verdadero si hay dos constantes $n_f$ y $c_f$ tal que para toda $n \geq n_f$,  $c_f g(n) \geq f(n)$.

\br La segunda, la afirmación de $g(n)$ es $\Omicron(h(n))$ es verdadera si hay dos constantes $n_g$ y $c_g$ tal que para toda $n \geq n_g$,  $c_g h(n) \geq g(n)$.

\br Ambas afirmaciones se realizan a partir de la definición de $\Omicron$.

\br El consecuente indica que $f(n)$ es $\Omicron(h(n))$ si hay hay dos constantes $n_h$ y $c_h$ tal que para toda $n \geq n_h$,  $c_h h(n) \geq f(n)$. $c_h$ debe ser igual a $c_f c_g$ y $n_h$ debe ser igual al valor mayor entre $n_f$ y $n_g$, o sea $\max \{n_f, n_g \}$.

\begin{propiedad}
  Si $f(n)$ es $\Omega(g(n))$ y $g(n)$ es $\Omega(h(n))$ $\Rightarrow$ $f(n)$ es $\Omega(h(n))$
\end{propiedad}

\br $f(n)$ es $\Omega(g(n))$ es verdadero si hay dos constantes $n_f$ y $c_f$ tal que para toda $n \geq n_f$,  $c_f g(n) \leq f(n)$.

\br La segunda, la afirmación de $g(n)$ es $\Omega(h(n))$ es verdadera si hay dos constantes $n_g$ y $c_g$ tal que para toda $n \geq n_g$,  $c_g h(n) \leq g(n)$.

\br Ambas afirmaciones se realizan a partir de la definición de $\Omega$.

\br El consecuente indica que $f(n)$ es $\Omega(h(n))$ si hay hay dos constantes $n_h$ y $c_h$ tal que para toda $n \geq n_h$,  $c_h h(n) \leq f(n)$. $c_h$ debe ser igual a $c_f c_g$ y $n_h$ debe ser igual al valor mayor entre $n_f$ y $n_g$, o sea $\max \{n_f, n_g \}$.

\begin{propiedad}
  Si $f(n)$ es $\Theta(g(n))$ y $g(n)$ es $\Theta(h(n))$ $\Rightarrow$ $f(n)$ es $\Theta(h(n))$
\end{propiedad}

\br $f(n)$ es $\Theta(g(n))$ si hay dos constantes $c_{1f}$ y $c_{2f}$ tales que $c_{1f}g(n) \geq f(n)$ (pertenencia a $\Omicron$) y $c_{2f}g(n) \leq f(n)$ (pertenencia a $\Omega$); así como un valor $n_f$ para el cual, para toda $n \geq n_f$, $c_{1f}g(n) \leq f(n)$ y $c_{2f}g(n) \geq f(n)$.

\br $g(n)$ es $\Theta(h(n))$ si hay dos constantes $c_{1g}$ y $c_{2g}$ tales que $c_{1g}h(n) \geq g(n)$ (pertenencia a $\Omicron$) y $c_{2g}h(n) \leq g(n)$ (pertenencia a $\Omega$); así como un valor $n_g$ para el cual, para toda $n \geq n_g$, $c_{1g}h(n) \geq g(n)$ y $c_{2g}h(n) \leq g(n)$.

\br $f(n)$ es $\Theta(h(n))$ si hay dos constantes $c_{1h}$ y $c_{2h}$, así como un valor $n_h$ para el cual $c_{1h}h(n) \geq f(n)$ (pertenencia a $\Omicron$) y $c_{2h}h(n) \leq f(n)$ (pertenencia a $\Omega$).

\br $c_{1h}$ tiene que ser igual a $c_{1f} \cdot c_{1g}$ para garantizar que $f(n)$ esté debajo de $g(n)$ y $h(n)$; mientras que $c_{1h}$ tiene que ser igual a $c_{2f} \cdot c_{2g}$ para garantizar que $f(n)$ esté encima de $g(n)$ y $h(n)$. $n_h$ es el valor mayor entre $n_f$ y $n_g$, ya que si a partir de $n_f$ la función siempre es mayor, cuando llegue $n_g$, que es el valor que cumple la condición, hará que ambas condiciones se cumplan.

\section{Reflexividad}

\begin{propiedad}
  $f(n)$ es $\Omicron(f(n))$
\end{propiedad}

\br Esta propiedad se cumple porque para pertenecer a $\Omicron$, hay un $c_f$ tal que $c_{f}f(n)) \geq f(n)$ para todo valor de $n$ a partir de cierto valor $n_f$. Al ser igual, se satisface la condición de "mayor o igual qué"; es decir, que existe un $c_f f(n) \geq f(n)$, el cual es igual a 1. $n_f$ puede ser cualquier valor de $n$, ya que siempre se cumplirá que $n_f \geq n$.

\begin{propiedad}
  $f(n)$ es $\Omega(f(n))$
\end{propiedad}

\br Esta propiedad se cumple porque para pertenecer a $\Omega$, hay un $c_f$ tal que $c_{f}f(n)) \leq f(n)$ para todo valor de $n$ a partir de cierto valor $n_f$. Al ser igual, se satisface la condición de "menor o igual qué"; es decir, que existe un $c_f f(n) \leq f(n)$, el cual es igual a 1. $n_f$ puede ser cualquier valor de $n$, ya que siempre se cumplirá que $n_f \geq n$.

\begin{propiedad}
  $f(n)$ es $\Theta(f(n))$
\end{propiedad}

\br Esta propiedad se cumple porque para pertenecer a $\Theta$, hay dos constantes $c_{1f}$ y $c_{2f}$ tal que $c_{1f}f(n)) \geq f(n)$ y $c_{2f}f(n)) \leq f(n)$ para todo valor de $n$ a partir de cierto valor $n_f$.

\br Al ser igual, se satisface la condición de "mayor o igual qué" para el caso de $c_{1f}$, es decir, que existe un $c_{1f} f(n) \geq f(n)$, el cual es igual a 1. También se satisface la condición de "menor o igual qué" para el caso de $c_{1f}$, es decir, que existe un $c_f f(n) \leq f(n)$, el cual es igual a 1. $n_f$ puede ser cualquier valor de $n$, ya que siempre se cumplirá que $n_f \geq n$.

\section{Simetría}
\begin{propiedad}
  $f(n)$ es $\Theta(g(n)) \Leftrightarrow$ $g(n)$ es $\Theta(f(n))$
\end{propiedad}

\br Para que $f(n)$ esté en $\Theta(g(n))$ se cumpla, tienen que existir 2 constantes $c_{1f}$ y $c_{2f}$ tales que $c_{1f}g(n) \geq f(n)$ y $c_{2f}g(n) \leq f(n)$ a partir de cierto valor de $n$ llamado $n_f$.

\br Después, para que $g(n)$ esté en $\Theta(f(n))$, tienen que existir 2 constantes $c_{1g}$ y $c_{2g}$ tales que $c_{1g}f(n) \geq g(n)$ y $c_{2g}f(n) \leq g(n)$ a partir de cierto valor de $n$ llamado $n_g$.

\br La elección de constantes que haría que se cumpla esta propiedad sería $c_{1f} = c_{2g}$ y $c_{2f} = c_{1g}$ y que $n_f = n_g$; es decir, la constante que dicta el límite superior para $\Theta(g(n))$ debe ser igual a la que define límite inferior para $\Theta(f(n))$, ya que las constantes desplazan a la función verticalmente, pero nunca la mueven de forma horizontal.

\section{Simetría Transpuesta}
\begin{propiedad}
  $f(n)$ es $\Omicron(g(n)) \Leftrightarrow$ $g(n)$ es $\Omega(f(n))$
\end{propiedad}

\br $f(n)$ es $\Omicron(g(n))$ si hay una constante $c_{f}$ tal que $c_f g(n) \geq f(n)$ para toda $n \geq n_{f}$.

\br Para que $g(n)$ sea $\Omega(f(n))$ tiene que haber una constante $c_g$ tal que $c_g f(n) \leq g(n)$ para toda $n \geq n_{g}$.

\br En este caso, las constantes que harían que se cumpla esta propiedad es que $c_{f}$ sea igual a $c_{g}$, ya que al pertenecer $f(n)$ a $\Omicron(g(n))$, siempre está debajo de $g(n)$, lo que desde la perspectiva de $f(n)$ indica que $g(n)$ siempre está encima de $f(n)$, confirmando la propiedad.

\section{Aditividad}

\begin{propiedad}
  $f(n)$ es $\Omicron(f(n))$ y $g(n)$ es $\Omicron(h(n))$ entonces $f(n) + g(n)$ es $\Omicron(h(n))$
\end{propiedad}

En este caso $f(n)$ siempre está por debajo o es igual a  $g(n)$ al pertenecer a $\Omicron$. Si $g(n)$ está debajo de $h(n)$ también, se puede recurrir a la propiedad de transitividad para indicar que $f(n)$ también está en $\Omicron(h(n))$. Al sumar $f(n)$ con $g(n)$, $f(n)$ individualmente a lo mucho puede adquirir el valor de $g(n)$, trasladando hacia arriba $g(n)$, pero sin pasar $h(n)$, ya que de lo contrario $g(n)$ ya no pertenecería a $\Omicron(h(n))$, entrando en una contradicción.

\begin{propiedad}
  $f(n)$ es $\Omega(g(n))$ y $g(n)$ es $\Omega(h(n))$ entonces $f(n) + g(n)$ es $\Omega(h(n))$
\end{propiedad}

En este caso $f(n)$ siempre está encima o es igual a  $g(n)$ al pertenecer a $\Omega$. Si $g(n)$ está encima de $h(n)$ también, se puede recurrir a la propiedad de transitividad para indicar que $f(n)$ también está en $\Omega(h(n))$. Al sumar $f(n)$ con $g(n)$, $f(n)$ individualmente a lo mucho puede adquirir el valor de $g(n)$, trasladando hacia abajo $g(n)$, pero sin pasar $h(n)$, ya que de lo contrario $g(n)$ ya no pertenecería a $\Omega(h(n))$, entrando en una contradicción.

\begin{propiedad}
  $f(n)$ es $\Theta(g(n))$ y $g(n)$ es $\Theta(h(n))$ entonces $f(n) + g(n)$ es $\Theta(h(n))$
\end{propiedad}

$f(n)$ siempre está entre el producto de $g(n)$, una que hace que esté debajo de $g(n)$ y otra que hace que esté encima de $g(n)$, esto por pertenecer a $\Theta$. Si $g(n)$ está entre $h(n)$ también, se puede recurrir a la propiedad de transitividad para indicar que $f(n)$ también está en $\Theta(h(n))$. Al sumar $f(n)$ con $g(n)$, $f(n)$ individualmente a lo mucho puede adquirir el valor de $g(n)$, trasladando verticalmente $g(n)$, pero sin pasar $h(n)$, ya que de lo contrario $g(n)$ ya no pertenecería a $\Theta(h(n))$, entrando en una contradicción.

\end{document}
