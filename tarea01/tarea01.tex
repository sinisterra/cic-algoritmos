\documentclass[hidelinks]{article}
\usepackage[spanish]{babel}
\usepackage{hyperref}
\newcommand\Omicron{O}
\newcommand{\br}{\vskip.2cm \noindent}

\newtheorem{propiedad}{Propiedad}[section]

\author{Santiago Sinisterra Sierra}
\title{Diseño y Análisis de Algoritmos \\ Tarea 1 - Demostraciones de las propiedades de los órdenes asintóticos}
\date{\today}

\begin{document}

\maketitle

\section{Transitividad}

\begin{propiedad}
  Si $f(n)$ es $\Omicron(g(n))$ y $g(n)$ es $\Omicron(h(n))$ $\Rightarrow$ $f(n)$ es $\Omicron(h(n))$
\end{propiedad}

\br El antecedente tiene dos partes. La primera, $f(n)$ es $\Omicron(g(n))$, es verdadera si hay dos constantes $n_f$ y $c_f$ tal que para toda $n \geq n_f$,  $c_f g(n) \geq f(n)$.

\br La segunda, la afirmación de $g(n)$ es $\Omicron(h(n))$ es verdadera si hay dos constantes $n_g$ y $c_g$ tal que para toda $n \geq n_g$,  $c_g h(n) \geq g(n)$.

\br Ambas afirmaciones se realizan a partir de la definición de $\Omicron$.

\br El consecuente indica que $f(n)$ es $\Omicron(h(n))$ si hay hay dos constantes $n_h$ y $c_h$ tal que para toda $n \geq n_h$,  $c_h h(n) \geq f(n)$. $c_h$ debe ser igual a $c_f c_g$ y $n_h$ debe ser igual al valor mayor entre $n_f$ y $n_g$, o sea $\max \{n_f, n_g \}$.

\begin{propiedad}
  Si $f(n)$ es $\Omega(g(n))$ y $g(n)$ es $\Omega(h(n))$ $\Rightarrow$ $f(n)$ es $\Omega(h(n))$
\end{propiedad}

\begin{propiedad}
  Si $f(n)$ es $\Theta(g(n))$ y $g(n)$ es $\Theta(h(n))$ $\Rightarrow$ $f(n)$ es $\Theta(h(n))$
\end{propiedad}

\section{Reflexividad}

\begin{propiedad}
  $f(n)$ es $\Omicron(f(n))$
\end{propiedad}

\begin{propiedad}
  $f(n)$ es $\Omega(f(n))$
\end{propiedad}

\begin{propiedad}
  $f(n)$ es $\Theta(f(n))$
\end{propiedad}

\section{Simetría}
\begin{propiedad}
  $f(n)$ es $\Theta(f(n)) \Leftrightarrow g(n) \Theta(f(n))$
\end{propiedad}

\section{Simetría Transpuesta}
\begin{propiedad}
  $f(n)$ es $\Omicron(f(n)) \Leftrightarrow g(n) \Omicron(f(n))$
\end{propiedad}

\section{Aditividad}
\begin{propiedad}
  $f(n)$ es $\Theta(f(n))$ y $g(n)$ es $\Theta(h(n))$ entonces $f(n) + g(n)$ es $\Theta(h(n))$
\end{propiedad}

\begin{propiedad}
  $f(n)$ es $\Omicron(f(n))$ y $g(n)$ es $\Omicron(h(n))$ entonces $f(n) + g(n)$ es $\Omicron(h(n))$
\end{propiedad}

\begin{propiedad}
  $f(n)$ es $\Omega(f(n))$ y $g(n)$ es $\Omega(h(n))$ entonces $f(n) + g(n)$ es $\Omega(h(n))$
\end{propiedad}


\end{document}
